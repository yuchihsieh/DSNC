\documentclass[11pt]{article}

% Standalone learning note (NOT part of the main paper).
\usepackage{amsmath,amssymb,amsthm}
\usepackage{mathtools}
\usepackage[hidelinks]{hyperref}
\usepackage{cleveref}
\usepackage{geometry}
\geometry{margin=1.7in}

% Theorem environments
\theoremstyle{definition}
\newtheorem{definition}{Definition}[section]
\newtheorem{remark}{Remark}[section]
\theoremstyle{plain}
\newtheorem{theorem}{Theorem}[section]
\newtheorem{lemma}{Lemma}[section]

\title{Notes: Continuous-time random matching (Duffie--Qiao--Sun 2025)}
\author{Yu-Chi Hsieh}
\date{\today}

\begin{document}
\maketitle
\tableofcontents

\newpage
\section{Purpose and a quick map to the paper}
\label{sec:dlq-purpose}

These notes summarize the setup and construction strategy of:
\begin{quote}
D.\ Duffie, L.\ Qiao, and Y.\ Sun (2025), \emph{Continuous time random matching}, The Annals of Applied Probability 35(3), 1755--1790.
\end{quote}
The goal is to understand (i) the continuous-time model objects, (ii) the deterministic ODE law for cross-sectional type distributions, and (iii) the probability-space/Fubini-extension construction used to make the model rigorous.

\medskip
\noindent\textbf{Where to look in DLQ (2025).} In the extracted text `paper\_text/aap\_2025\_dqs\_extracted.txt`:
\begin{itemize}
  \item Model primitives and objects $(p_0,\eta,\theta,\varsigma)$, definition of $R(p)$, and ODE: around ``Model and results'' and equation (2).
  \item Formal model definition (random type process $\alpha$ and matching process $\varphi$, measurability, RCLL, etc.): \textbf{Definition 1}.
  \item Main existence/properties theorem: \textbf{Theorem 1}.
  \item Nonstandard machinery overview: \textbf{Section 4}.
  \item The ``generalized Keisler'' ingredient (Loeb transition probabilities and generalized Fubini): \textbf{Section 4.4 and Section 5}, with explicit reference to Duffie--Sun (2007) for the generalized Fubini theorem for Loeb transition probabilities.
\end{itemize}

\section{Minimal background: CTMCs, generators, and intensities (as used in DLQ)}
\label{sec:ctmc-background}

DLQ's model is built so that each agent's type evolves as a \emph{continuous-time Markov chain} (CTMC) on a finite state space $S=\{1,\dots,K\}$. We record the minimal definitions used in their statements.

\begin{definition}[Generator / intensity matrix on a finite state space]
Let $S=\{1,\dots,K\}$. A \emph{generator} (or \emph{intensity matrix}) is a matrix $R=(R_{kr})_{k,r\in S}$ such that
\begin{itemize}
  \item $R_{kr}\ge 0$ for $k\neq r$,
  \item $R_{kk}=-\sum_{r\neq k}R_{kr}$ (row sums equal $0$).
\end{itemize}
For a CTMC $X(t)$ on $S$ with generator $R$, the instantaneous transition rate from state $k$ to $r\neq k$ at time $t$ is $R_{kr}$.
\end{definition}

\begin{remark}[Kolmogorov forward equation]
If $\pi(t)$ is the row vector of state probabilities $\pi_k(t)=\mathbb P(X(t)=k)$, then $\pi(t)$ solves the ODE
\[
\frac{d}{dt}\pi(t)=\pi(t)R.
\]
DLQ uses a version of this equation with a \emph{state-dependent} generator $R(p_t)$ that depends on the cross-sectional distribution $p_t$.
\end{remark}

\begin{remark}[Counting processes and martingale characterization (informal)]
DLQ encodes ``$k\to r$ transitions occur at intensity $R_{kr}(p_t)$'' via counting processes $C_{ikr}(t)$ and compensated processes
\[
M_{ikr}(t)=C_{ikr}(t)-\int_0^t \mathbf 1\{\alpha(i,s)=k\}R_{kr}(p_s)\,ds,
\]
which are (local) martingales. You do not need deep martingale theory for the big picture; this is mainly a standard way to define/verify intensities.
\end{remark}

\section{DLQ (2025) model: primitives, processes, and the ODE}
\label{sec:dlq-model}

\subsection{Agent space, sample space, and Fubini extension}
DLQ assumes an atomless agent space $(I,\mathcal I,\lambda)$ with $\lambda(I)=1$ and a probability space $(\Omega,\mathcal F,P)$ for randomness. Because joint measurability and independence are incompatible on the usual product, they work on a \emph{Fubini extension}
\[
(I\times\Omega,\ \mathcal I\boxtimes\mathcal F,\ \lambda\boxtimes P),
\]
in the sense of Sun (2006, Definition 2.2) (DLQ repeats the definition early in Section 2).

\subsection{Time domain and measurability conventions}
\label{subsec:dlq-time-domain}
DLQ's time domain is $\mathbb R_+=[0,\infty)$ equipped with its Borel $\sigma$-algebra $\mathcal B(\mathbb R_+)$.
The model's primitive objects and solution objects are formulated on the \emph{triple} product
\[
I\times \Omega \times \mathbb R_+,
\]
with measurability taken with respect to the product $\sigma$-algebra $(\mathcal I\boxtimes\mathcal F)\otimes \mathcal B(\mathbb R_+)$.

\begin{remark}[Which measure do we use on $\mathbb R_+$?]
There are two distinct roles of time:
\begin{itemize}
  \item \textbf{Measurability/topology:} for processes like $\alpha(i,\omega,t)$ and $\varphi(i,\omega,t)$ one only needs the Borel $\sigma$-algebra and path regularity (RCLL/c\`adl\`ag).
  \item \textbf{Integration over time:} whenever DLQ writes $\int_0^t(\cdots)\,ds$, the measure is the \emph{Lebesgue measure} $ds$ on $\mathbb R_+$.
\end{itemize}
Since Lebesgue measure on $\mathbb R_+$ is $\sigma$-finite, all time-integrals can be understood locally on each finite horizon $[0,T]$ and then extended by letting $T\uparrow\infty$.
This is also the natural setup for DSNC: prove measurability/LLN/ODE claims on finite horizons and then take an increasing limit.
\end{remark}

\subsection{Primitives}
Fix $K\ge 2$ and let $S=\{1,\dots,K\}$ and let $\Delta$ be the simplex of probability measures on $S$.
DLQ's primitives are:
\begin{itemize}
  \item An initial cross-sectional type distribution $p_0\in\Delta$.
  \item Mutation intensities $\eta_{kl}\in[0,\infty)$ for $k\neq l$.
  \item Matching intensities $\theta_{kl}:\Delta\to\mathbb R_+$ satisfying mass-balance
  \[
  p_k\theta_{kl}(p)=p_l\theta_{lk}(p),
  \]
  and a Lipschitz condition on $p_k\theta_{kl}(p)$ (DLQ uses this to ensure well-posedness of the ODE).
  \item Match-induced type-change distributions $\varsigma_{kl}\in\Delta$ (the new type distribution for a type-$k$ agent matched to a type-$l$ agent).
\end{itemize}

\subsection{\texorpdfstring{The induced generator $R(p)$}{The induced generator R(p)} and the deterministic ODE}
Define $R(p)$ (a generator on $S$) by combining mutation and match-induced type change:
\[
R_{kr}(p)=\eta_{kr}+\sum_{l=1}^K \theta_{kl}(p)\,\varsigma_{kl}(r),\qquad k\neq r,
\quad
R_{kk}(p)=-\sum_{r\neq k}R_{kr}(p).
\]
DLQ's key deterministic law is that the (expected, and in fact realized) cross-sectional type distribution solves
\[
\frac{d}{dt}x(t)=x(t)R(x(t)).
\]

\section{Continuous-time matching objects (DLQ Definition 1) and main result (Theorem 1)}
\label{sec:dlq-def1-thm1}

DLQ defines two coupled processes on $I\times\Omega\times\mathbb R_+$:
\begin{itemize}
  \item a \emph{type process} $\alpha:I\times\Omega\times\mathbb R_+\to S$ (RCLL in $t$),
  \item a \emph{matching process} $\varphi:I\times\Omega\times\mathbb R_+\to I$ (RCLL in $t$),
\end{itemize}
such that for each fixed $(\omega,t)$ the map $i\mapsto \varphi(i,\omega,t)$ is an involution (pairwise matching) and is measure-preserving in $i$ for $\lambda$.

\begin{remark}[Measurability level for the processes (as in DLQ Definition 1)]
DLQ's Definition 1 imposes $(\mathcal I\boxtimes\mathcal F)\otimes \mathcal B(\mathbb R_+)$-measurability for $\alpha$ (and appropriate measurability for $\varphi$), and RCLL/c\`adl\`ag path regularity in $t$ for almost every agent/sample point.
For each fixed $t$, the section $\varphi_t(\cdot,\cdot)\coloneqq \varphi(\cdot,\cdot,t)$ is a random matching on the Fubini extension $(I\times\Omega,\mathcal I\boxtimes\mathcal F,\lambda\boxtimes P)$: for $P$-a.e.\ $\omega$, the map $i\mapsto \varphi(i,\omega,t)$ is a measure-preserving involution.
\end{remark}

\begin{remark}[Interpretation of $\varphi(i,\omega,t)$]
DLQ's $\varphi(i,\omega,t)=j$ means: as of time $t$ in state $\omega$, agent $j$ is the \emph{last} partner met by $i$ (and symmetrically $i$ is the last partner met by $j$), with no intervening meetings since that last meeting time.
\end{remark}

\medskip
\noindent DLQ's \textbf{Theorem 1} asserts existence of such a system on some Fubini extension, and that:
\begin{itemize}
  \item the realized cross-sectional distribution $p_t(\omega)$ is almost surely equal to its expectation and solves the ODE,
  \item the cross-sectional type process is a Markov chain with generator $R(\bar p_t)$,
  \item match counts aggregate deterministically at rate $\bar p_{t,k}\theta_{kl}(\bar p_t)$,
  \item there exist stationary distributions $p^\ast$ satisfying $p^\ast R(p^\ast)=0$.
\end{itemize}

\section{How DLQ constructs the model (roadmap)}
\label{sec:dlq-roadmap}

DLQ's existence proof is long because \emph{continuous time} forces you to control the cumulative effect of small dependence across \emph{many} periods.
Here is a more detailed roadmap than the one-page overview typically given in summaries.

\subsection{Layer 1: a finite-agent, small-step discrete-time model (DLQ Section 3; proofs in the supplement)}
\label{subsec:dlq-roadmap-finite}
Fix a large integer $M$, a finite population of $\bar M$ agents, and a finite type space $S=\{1,\dots,K\}$.
DLQ constructs a discrete-time model on a fine grid with step size on the order of $1/M$.
Within each ``period'' there are three sequential stages:
\begin{itemize}
  \item \textbf{Mutation step:} each type-$k$ agent mutates to type $l$ with probability $O(1/M)$ so that mutation becomes a Poisson intensity $\eta_{kl}$ in the limit.
  \item \textbf{Matching step:} conditional on the current cross-sectional distribution, each type-$k$ agent meets a type-$l$ partner with probability $O(1/M)$, scaled to approximate the intended matching intensity $\theta_{kl}(p)$.
  \item \textbf{Match-induced type change:} conditional on partner types $(k,l)$, an agent's post-meeting type is drawn from $\varsigma_{kl}$.
\end{itemize}

\medskip
\noindent\textbf{What must be proved in the finite model.}
Because matching is \emph{without replacement} and repeats over many periods, partner draws are not independent across agents, and dependence can accumulate across time.
DLQ therefore proves a collection of quantitative lemmas in Section 3, with proofs in the supplement:
\begin{itemize}
  \item \textbf{Within-period matching accuracy:} the realized type-by-type matching frequencies are close to their targets (up to vanishing rounding errors).
  \item \textbf{Approximate conditional independence / approximate Markov property:} one-step conditional laws of finite-agent observables factorize up to small errors.
  \item \textbf{Cumulative error control across many steps:} Appendix A provides general lemmas for controlling the probability of multiple ``rare'' events over many steps; the supplement applies them to the model's jump counts and dependence terms.
\end{itemize}
This is the technical core: the errors must be small enough that when you later take a hyperfinite limit (infinitely many steps), they still wash out.

\subsection{Layer 2: transfer to a hyperfinite time grid (DLQ Section 5)}
\label{subsec:dlq-roadmap-hyperfinite}
Replace the large integer $M$ by an \emph{unlimited hyperinteger} and transfer the entire finite construction to obtain an \emph{internal} model on:
\begin{itemize}
  \item a hyperfinite agent set $I$ with internal counting measure $\lambda_0$, and
  \item a hyperfinite time grid with infinitesimal step length $1/M$.
\end{itemize}
At this stage, DLQ has an internally well-defined discrete-time system whose one-step transition probabilities encode the intended continuous-time intensities, up to infinitesimal errors.

\subsection{Layer 3: Loeb-ization + generalized Fubini + exact LLN (DLQ Section 4.4; DS07 Section 5; Sun 2006)}
\label{subsec:dlq-roadmap-loeb}
Convert the internal (hyperfinite) probability objects into standard ones via Loeb measures.
The key technical ingredients are:
\begin{itemize}
  \item \textbf{Generalized Fubini for Loeb transition probabilities (DLQ Section 4.4):} this yields a joint agent--sample measure even when the sample law depends on an agent/state via a kernel.
  \item \textbf{Infinite kernel products (generalized Ionescu--Tulcea) for Loeb transition probabilities (Duffie--Sun 2007, Section 5):} this constructs a path-space law from a sequence of kernels.
  \item \textbf{Exact LLN for stochastic processes on a Fubini extension (Sun 2006, Theorem 2.16):} once essential pairwise independence is obtained, cross-sectional paths become deterministic a.s., and the deterministic limit solves the ODE.
\end{itemize}

\section{The ``generalized Keisler'' ingredient: Loeb transition probabilities}
\label{sec:dlq-generalized-keisler}

Continuous time (built from hyperfinite discrete time) naturally leads to an \emph{iterated} randomization across periods.
This is cleanly expressed using \emph{transition probabilities} (kernels) rather than a single product measure.
DLQ therefore uses the Loeb-transition-probability technology developed in Duffie--Sun (2007, Section 5).

\subsection{Internal transition probabilities and their Loeb product (DLQ Section 4.4)}
\label{subsec:dlq-loeb-transition}
We keep this section at the level needed for DLQ, and refer to the static DS07 note for nonstandard basics (internal sets, Loeb measures, etc.).

\begin{definition}[Internal transition probability]
Let $(I,\mathcal I_0,\lambda_0)$ be a hyperfinite internal probability space where $\mathcal I_0$ is the internal power set on a hyperfinite set $I$.
Let $\Omega$ be a hyperfinite internal set with internal power set $\mathcal F_0$.
An \emph{internal transition probability} is an internal function $P_0$ that assigns to each $i\in I$ a hyperfinite internal probability measure $P_{0i}$ on $(\Omega,\mathcal F_0)$.
\end{definition}

\begin{definition}[Internal product transition probability]
Given $\lambda_0$ and an internal transition probability $P_0=\{P_{0i}\}_{i\in I}$, define an internal probability measure $\tau_0$ on $(I\times\Omega,\ \mathcal I_0\otimes\mathcal F_0)$ by
\[
\tau_0(\{(i,\omega)\}) \coloneqq \lambda_0(\{i\})\,P_{0i}(\{\omega\}).
\]
\end{definition}

Let $\lambda$ be the Loeb measure associated with $\lambda_0$, let $P_i$ be the Loeb measure associated with $P_{0i}$, and let $\tau$ be the Loeb measure associated with $\tau_0$.
The collection $P=\{P_i\}_{i\in I}$ is called a \emph{Loeb transition probability}, and $\tau$ is the \emph{Loeb product transition probability} of $\lambda$ and $P$ (DLQ's terminology).
DLQ also denotes $\tau$ by $\lambda\boxtimes P$.

\subsection{Generalized Fubini theorem (DLQ Proposition 4; proved in Duffie--Sun 2007)}
\label{subsec:dlq-generalized-fubini}
\begin{theorem}[Generalized Fubini for a Loeb transition probability (DLQ Proposition 4, informal)]
Let $f$ be a real-valued $\tau$-integrable function on $(I\times\Omega,\ \sigma(\mathcal I_0\otimes \mathcal F_0),\ \tau)$.
Then:
\begin{enumerate}
  \item for $\lambda$-a.e.\ $i$, the section $f_i(\cdot)=f(i,\cdot)$ is $\sigma(\mathcal F_0)$-measurable and integrable on $(\Omega,\sigma(\mathcal F_0),P_i)$;
  \item the function $i\mapsto \int_\Omega f_i(\omega)\,dP_i(\omega)$ is integrable on $(I,\sigma(\mathcal I_0),\lambda)$;
  \item the Fubini identity holds:
  \[
  \int_{I\times\Omega} f(i,\omega)\,d\tau(i,\omega)
  =
  \int_I\left(\int_\Omega f(i,\omega)\,dP_i(\omega)\right)\,d\lambda(i).
  \]
\end{enumerate}
\end{theorem}

\begin{remark}[Why this is needed beyond Keisler (two-factor) Fubini]
In static matching, the sample measure on $\Omega$ does not depend on $i$, and Keisler's Loeb-product Fubini theorem suffices. In dynamic/hyperfinite-time constructions, the period-$n$ randomization may depend on past states (hence on the current ``state'' variable), leading naturally to transition kernels. One then needs a Fubini theorem compatible with \emph{kernel products}, and eventually an Ionescu--Tulcea type theorem to build the product/path measure.
\end{remark}

\begin{remark}[How this ensures existence of a suitable extension (big picture)]
For a \emph{single} kernel $P=\{P_i\}$, the generalized Fubini theorem above ensures that $\lambda\boxtimes P$ behaves like a ``product'' measure with respect to iterated integration, even though the second factor depends on the first.

For \emph{dynamic} matching, one needs an \emph{infinite} composition of kernels (one per randomization step) to construct a path-space probability measure.
Duffie--Sun (2007, Section 5) proves a generalized Ionescu--Tulcea theorem for Loeb transition probabilities and shows that the resulting infinite product yields a Fubini extension on the joint agent--path space.
This is the core existence machinery DLQ calls ``generalized Keisler.''
\end{remark}

\section{Connections to Duffie--Sun (2007) and Sun (2006)}
\label{sec:dlq-connections}

DLQ (2025) is continuous-time, but the logic is consistent with the earlier discrete-time/static program:
\begin{itemize}
  \item Duffie--Sun (2007) provides the Loeb-space construction for static/discrete-time matching and develops generalized Fubini/Ionescu--Tulcea type results for Loeb transition probabilities (their Section 5), which DLQ cites and builds on.
  \item Sun (2006) provides the Fubini extension framework and the exact LLN for a continuum of stochastic processes (Theorem 2.16), which DLQ invokes to pass from independent agent-level dynamics to deterministic cross-sectional dynamics.
\end{itemize}

\section{What to read next in DLQ (2025)}
\label{sec:dlq-next}

For a first deep read focused on construction:
\begin{itemize}
  \item \textbf{Section 2}: formal definition of the model objects and Fubini extension.
  \item \textbf{Section 3}: finite-agent discrete approximation (where the technical estimates live; many proofs are in the supplement).
  \item \textbf{Section 4.4 and Section 5}: Loeb transition probabilities and the hyperfinite-time-to-continuous-time construction (the ``generalized Keisler'' part).
  \item \textbf{Sun (2006), Theorem 2.16}: exact LLN for stochastic processes, used to deduce that cross-sectional paths are deterministic almost surely.
\end{itemize}

\end{document}


