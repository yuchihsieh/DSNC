\documentclass[11pt]{article}

% Standalone note (NOT part of the main paper).
\usepackage{amsmath,amssymb,amsthm}
\usepackage{mathtools}
\usepackage[hidelinks]{hyperref}
\usepackage{cleveref}
\usepackage{geometry}
\geometry{margin=2.5in}

% Theorem environments
\theoremstyle{definition}
\newtheorem{definition}{Definition}[section]
\newtheorem{remark}{Remark}[section]
\theoremstyle{plain}
\newtheorem{theorem}{Theorem}[section]

\title{Notes: Fubini extensions for discrete-time random matching dynamics\\(Duffie--Sun 2007, Section 3)}
\author{Yu-Chi Hsieh}
\date{\today}

\begin{document}
\maketitle
\tableofcontents

\newpage
\section{Discrete-time random matching dynamics (Duffie--Sun 2007, Section 3)}
\label{sec:ds2007-discrete}

\subsection{Quick map to the source papers}
\label{subsec:quick-map-ds2007-discrete}
\noindent
This note is distilled primarily from:
\begin{itemize}
  \item D.\ Duffie and Y.\ Sun (2007), \emph{Existence of independent random matching} (AAP 17), especially \textbf{Section 3} and the existence proof via \textbf{Sections 5--6}, and
  \item Y.\ Sun (2006), \emph{The exact law of large numbers via Fubini extension} (JET 126), for the ELLN that turns essential pairwise independence into deterministic aggregation.
\end{itemize}

\medskip
\noindent
\textbf{Where to look in DS2007 (as extracted in this repo).}
\begin{itemize}
  \item \textbf{Dynamic model definition (three-stage per period):} DS2007 \S3.1, starting around the formal setup of $a_0$, $h_n$, $\pi_n$, $g_n$, $a_n$.
  \item \textbf{Markov conditional independence in types:} DS2007 \S3.2, equations (5)--(7).
  \item \textbf{Existence theorem:} DS2007 \S3.3, \textbf{Theorem 3.1}.
  \item \textbf{Technical construction:} DS2007 \S5 (Loeb transition probabilities, generalized Fubini, generalized Ionescu--Tulcea) and \S6 (proof of Theorem 3.1).
\end{itemize}

\subsection{Aim and scope}
\label{subsec:aim-scope-ds2007-discrete}
The goal is to formalize a standard modeling template used in economics: a continuum of agents repeatedly undergo
\begin{enumerate}
  \item idiosyncratic type changes (``mutation''),
  \item random (partial) matching, and
  \item match-induced type changes,
\end{enumerate}
and to show that such a system \emph{exists} on a probability space that avoids the usual measurability paradoxes, and that (under appropriate independence) the realized cross-sectional type distribution evolves \emph{deterministically} via an exact law of large numbers (ELLN).

This note focuses on the \emph{discrete-time} system of DS2007. It does \emph{not} redo the full nonstandard proof, but it states the model and independence conditions precisely and explains the construction strategy and what it buys you.

\subsection{Measure-theoretic ambient space: Fubini extensions}
\label{subsec:fubini-extension-discrete}
We follow Sun (2006) / Duffie--Sun (2007) and work on a Fubini extension.

\begin{definition}[Fubini extension (Sun 2006, Def.\ 2.2; DS2007, Def.\ 2.1)]
Let $(I,\mathcal I,\lambda)$ and $(\Omega,\mathcal F,P)$ be probability spaces. A probability space
\[
(I\times\Omega,\ \mathcal I\boxtimes\mathcal F,\ \lambda\boxtimes P)
\]
is a \emph{Fubini extension} of the usual product $(I\times\Omega,\mathcal I\otimes\mathcal F,\lambda\otimes P)$ if:
\begin{enumerate}
  \item $\mathcal I\otimes\mathcal F\subseteq \mathcal I\boxtimes\mathcal F$ and $\lambda\boxtimes P$ extends $\lambda\otimes P$; and
  \item every $\lambda\boxtimes P$-integrable $f:I\times\Omega\to\mathbb R$ satisfies the usual Fubini equalities.
\end{enumerate}
\end{definition}

\begin{remark}[Why we work on $\boxtimes$ rather than $\otimes$]
On the usual product $\sigma$-algebra $\mathcal I\otimes\mathcal F$, joint measurability and a continuum of (essentially) independent random variables are incompatible except in degenerate cases (Sun 2006, Prop.\ 2.1). The Fubini extension $\mathcal I\boxtimes\mathcal F$ enlarges measurability while retaining Fubini.
\end{remark}

\subsection{Model primitives and objects (DS2007, Section 3.1)}
\label{subsec:ds2007-discrete-model}

\subsubsection{Agents, types, time}
Fix a finite type space
\[
S=\{1,2,\dots,K\}.
\]
Time is discrete: $n=0,1,2,\dots$.

Let $(I,\mathcal I,\lambda)$ be an atomless probability space of agents, and $(\Omega,\mathcal F,P)$ a probability space of randomization.
Work on a Fubini extension $(I\times\Omega,\mathcal I\boxtimes\mathcal F,\lambda\boxtimes P)$.

\subsubsection{State variables (type processes) and per-period randomizations}
DS2007 represent the cross-sectional state via \emph{type functions} (end-of-period types) and intermediate-stage type functions.

\begin{definition}[Type functions and stages (DS2007 notation)]
An \emph{initial type function} is an $\mathcal I$-measurable map
\[
a_0:I\to S,
\]
with induced initial cross-sectional distribution $p^0\in\Delta(S)$ given by $p^0(k)=\lambda(a_0^{-1}(\{k\}))$.

For each period $n\ge 1$, define:
\begin{itemize}
  \item \textbf{Mutation-stage type} $h_n:I\times\Omega\to S$ (type \emph{after} mutation, before matching).
  \item \textbf{Partial matching map} $\pi_n:I\times\Omega\to I\cup\{J\}$, where $J$ denotes ``no match''.
  \item \textbf{Partner-type process} $g_n:I\times\Omega\to S\cup\{J\}$ given by
  \[
  g_n(i,\omega)\coloneqq h_n(\pi_n(i,\omega),\omega),
  \]
  with the convention $h_n(J,\omega)=J$ (so $g_n(i,\omega)=J$ if $i$ is unmatched).
  \item \textbf{End-of-period type} $a_n:I\times\Omega\to S$ (type after the match-induced type change).
\end{itemize}
The sequence $(a_n)_{n\ge 0}$ is the primary state process (types at the end of each period).
\end{definition}

\begin{remark}[``Rematching'' and uniqueness of partners]
Each period has a new draw of $\pi_n$, so agents can be matched in many different periods (rematching over time). Within a fixed period $n$ each agent is either unmatched or matched to exactly one partner (a partial matching).
\end{remark}

\subsubsection{Parameters (primitives) \texorpdfstring{$(p^0,b,q,v)$}{(p0,b,q,v)}}
DS2007 take as given:
\begin{itemize}
  \item an initial cross-sectional distribution $p^0\in\Delta(S)$;
  \item a mutation transition matrix $b=(b_{k\ell})_{k,\ell\in S}$, where $b_{k\ell}\in[0,1]$ and $\sum_{\ell} b_{k\ell}=1$ for each $k$;
  \item no-match probabilities $q=(q_k)_{k\in S}$ with $q_k\in[0,1]$;
  \item match-induced type-change kernels $v=(v_{k\ell})_{k,\ell\in S}$, where for each $(k,\ell)$, $v_{k\ell}$ is a probability distribution on $S$ (write $v_{k\ell}(r)$ for the probability of new type $r$ for a type-$k$ agent matched with a type-$\ell$ agent).
\end{itemize}

\subsubsection{Per-stage distributional requirements (DS2007, Section 3.1)}
DS2007 impose stage-by-stage conditional laws. Informally:
\begin{itemize}
  \item \textbf{Mutation:} conditional on $a_{n-1}(i,\cdot)=k$, the post-mutation type $h_n(i,\cdot)$ has law $b_{k\cdot}$.
  \item \textbf{Partial matching:} conditional on $h_n(i,\cdot)=k$, the event ``unmatched'' has probability $q_k$, and if matched, the partner-type distribution is proportional to the current post-mutation cross-sectional type masses among those who are matched (DS2007 equation (3)).
  \item \textbf{Type change:} conditional on $h_n(i,\cdot)=k$ and partner type $g_n(i,\cdot)=\ell$ (or $J$), the end-of-period type $a_n(i,\cdot)$ is distributed according to $v_{k\ell}$ (or stays at $k$ if unmatched), as in DS2007 equation (4).
\end{itemize}

\begin{remark}[Expected post-mutation distribution]
DS2007 define the \emph{expected} post-mutation cross-sectional distribution by
\[
\bar p_{n-\frac12}(k)\coloneqq \int_\Omega \lambda(\{i\in I:\ h_n(i,\omega)=k\})\,dP(\omega),
\]
which only depends on primitives $(p^0,b,q,v)$.

Under suitable independence (see below), Sun-style ELLN implies the \emph{realized} post-mutation mass $\lambda(\{i:h_n(i,\omega)=k\})$ equals $\bar p_{n-\frac12}(k)$ for $P$-a.e.\ $\omega$.
\end{remark}

\subsection{Independence: Markov conditional independence in types (DS2007, Section 3.2)}
\label{subsec:markov-conditional-independence}

The key condition is designed to ensure that each per-period stage is (essentially) independent across agents \emph{given the appropriate Markov state}, while allowing the across-time dependence that defines Markov chains.

\begin{definition}[Markov conditional independence in types (DS2007, equations (5)--(7), informal)]
Fix $n\ge 1$. The dynamical system is \emph{Markov conditionally independent in types} at time $n$ if, for $\lambda$-a.e.\ $i$ and $\lambda$-a.e.\ $j$:
\begin{itemize}
  \item \textbf{(Mutation step)} conditional on the current end-of-period types $a_{n-1}(i,\cdot)$ and $a_{n-1}(j,\cdot)$ (and the past), the random variables $h_n(i,\cdot)$ and $h_n(j,\cdot)$ are independent.
  \item \textbf{(Partial matching step)} conditional on post-mutation types $h_n(i,\cdot)$ and $h_n(j,\cdot)$ (and the past and mutation outcomes), the random variables $g_n(i,\cdot)$ and $g_n(j,\cdot)$ are independent.
  \item \textbf{(Type-changing step)} conditional on $(h_n(i,\cdot),g_n(i,\cdot))$ and $(h_n(j,\cdot),g_n(j,\cdot))$ (and the past), the random variables $a_n(i,\cdot)$ and $a_n(j,\cdot)$ are independent.
\end{itemize}
Moreover, each step is required to depend on the past only through the relevant Markov state(s) (the DS2007 conditioning $\sigma$-fields encode this precisely).
\end{definition}

\begin{remark}[Why this is the right condition for exact aggregation]
Sun (2006) gives ELLN statements for essentially pairwise independent processes (and conditional variants). DS2007's condition is tailored so that, at each stage, the relevant cross-sectional quantities aggregate deterministically, and the end-of-period type process becomes a continuum of independent Markov chains.
\end{remark}

\subsection{Main existence theorem (DS2007, Theorem 3.1)}
\label{subsec:theorem-3-1}

\begin{theorem}[Existence of discrete-time matching dynamics (Duffie--Sun 2007, Thm.\ 3.1)]
Fix parameters $(p^0,b,q,v)$ as above. There exists a Fubini extension
\[
(I\times\Omega,\mathcal I\boxtimes\mathcal F,\lambda\boxtimes P)
\]
of the usual product probability space and a dynamical system $(a_n,h_n,\pi_n,g_n)_{n\ge 1}$ with initial type function $a_0$ such that:
\begin{itemize}
  \item the mutation, partial matching, and match-induced type change steps have the prescribed conditional laws given by $(b,q,v)$; and
  \item the system is Markov conditionally independent in types (in the DS2007 sense).
\end{itemize}
\end{theorem}

\subsection{How the existence proof works (very high-level)}
\label{subsec:proof-scheme}
DS2007 prove the theorem using nonstandard analysis and Loeb measures.

\begin{itemize}
  \item \textbf{Hyperfinite agents:} start with a hyperfinite agent set $I=\{1,\dots,M\}$, with internal counting probability $\lambda_0(A)=|A|/M$, then Loeb-ize to get an atomless standard agent space $(I,\mathcal I,\lambda)$.

  \item \textbf{Stage-by-stage internal transition probabilities:} each period has three randomization stages. DS2007 build these using internal transition probabilities (kernels) on hyperfinite spaces:
  \begin{itemize}
    \item mutation draws are built from a hyperfinite product space whose coordinate functions are \emph{internally independent} (transfer of finite-product independence),
    \item matching draws use a hyperfinite space of partial matchings (extending the static perfect-matching construction), and
    \item type-change draws are again coordinatewise products conditional on the realized post-mutation and partner types.
  \end{itemize}

  \item \textbf{Infinite-horizon gluing:} to combine all periods into one global sample space, DS2007 construct an infinite product of Loeb transition probabilities. The technical spine is:
  \begin{itemize}
    \item a generalized Fubini theorem for Loeb transition probabilities (DS2007 Theorem 5.1),
    \item a generalized Ionescu--Tulcea theorem for the infinite product system (DS2007 Theorem 5.5),
    \item and the resulting Fubini extension on the infinite product space (DS2007 Proposition 5.9).
  \end{itemize}
  This yields a standard probability space that supports the entire discrete-time dynamical system and preserves the Fubini property needed for iterated integrals and ELLN arguments.
\end{itemize}

\subsection{What you get after existence: deterministic aggregates via ELLN}
\label{subsec:elln-implications-discrete}
Once the relevant processes are essentially pairwise independent on a Fubini extension (or conditionally so, at each stage), Sun (2006) implies exact aggregation:
\begin{itemize}
  \item realized cross-sectional distributions (of $h_n$, $g_n$, $a_n$) coincide almost surely with their expectations;
  \item the cross-sectional distribution of $a_n$ is deterministic and evolves according to a recursion determined by the micro transition primitives $(b,q,v)$ and the matching rule;
  \item individual type paths $(a_n(i,\cdot))_{n\ge 0}$ form a continuum of independent Markov chains (in the precise DS2007/companion-paper sense).
\end{itemize}

\subsection{Bibliographic notes}
\label{subsec:biblio}
\begin{quote}
D.\ Duffie and Y.\ Sun, \emph{Existence of independent random matching}, Annals of Applied Probability 17 (2007), 386--419.
\end{quote}
\begin{quote}
Y.\ Sun, \emph{The exact law of large numbers via Fubini extension and characterization of insurable risks}, Journal of Economic Theory 126 (2006), 31--69.
\end{quote}

\appendix

\section{Appendix: Hyperfinite, stage-by-stage construction (DS2007, Section 6 ``idea level'')}
\label{app:hyperfinite-stage-by-stage}

This appendix makes the stage-wise construction explicit on a \emph{hyperfinite} agent set, in the style of DS2007 Section~6. The goal is to see concretely how one produces
\[
(h_n,\pi_n,g_n,a_n) \quad\text{from}\quad (a_{n-1}; b,q,v)
\]
at the internal level, so that after Loeb-ization the model satisfies the desired conditional laws and (essential) conditional independence.

\subsection{Hyperfinite agent set and internal counting measure}
\label{app:hyperfinite-agents}
Fix an unlimited hyperinteger $M\in{}^\ast\mathbb N$ and let
\[
I\coloneqq\{1,2,\dots,M\}\subset{}^\ast\mathbb N.
\]
Let $\mathcal I_0$ be the internal power set of $I$ (internal subsets), and define the internal counting probability measure
\[
\lambda_0(A)\coloneqq \frac{|A|}{M},\qquad A\in\mathcal I_0.
\]
Loeb-izing yields the standard (atomless) agent space $(I,\mathcal I,\lambda)$.

\subsection{How DS2007 organizes time: three internal stages per period}
\label{app:three-stages-per-period}
DS2007's Section~6 indexes the randomness as an infinite sequence of internal sample spaces. For each period $n\ge 1$, there are three stages, which we can label as:
\[
\text{mutation stage }(3n-2),\qquad \text{matching stage }(3n-1),\qquad \text{type-change stage }(3n).
\]
At each stage, the random object is typically an internal function $\omega:I\to(\cdot)$ on the agent set, so that agent-level randomness is obtained by evaluation at $i$.

\subsection{Mutation stage: product draw on $S^I$}
\label{app:mutation-stage}
\noindent\textbf{Input:} end-of-period type profile $a_{n-1}:I\to S$ (internal at the hyperfinite level).

\medskip
\noindent\textbf{Goal:} construct an internal random post-mutation profile $h_n:I\to S$ such that for each agent $i$ and types $k,\ell\in S$,
\[
P(h_n(i)=\ell\mid a_{n-1}(i)=k)=b_{k\ell},
\]
and such that (conditionally on $a_{n-1}$) the family $\{h_n(i)\}_{i\in I}$ is internally independent.

\medskip
\noindent\textbf{Construction:}
Let $S^I$ denote the internal set of all internal functions $\omega:I\to S$, with internal power set $\mathcal F^{(3n-2)}_0$.
For each agent $i\in I$, define the (internal) probability vector on $S$
\[
\gamma_i(\ell)\coloneqq b_{a_{n-1}(i),\ell}.
\]
Define an internal probability measure $Q^{(3n-2)}_0$ on $S^I$ by the hyperfinite product
\[
Q^{(3n-2)}_0 \coloneqq \bigotimes_{i\in I}\gamma_i.
\]
Then draw $\omega^{(3n-2)}\sim Q^{(3n-2)}_0$ and set
\[
h_n(i)\coloneqq \omega^{(3n-2)}(i).
\]
Because $Q^{(3n-2)}_0$ is a (hyperfinite) product measure, the coordinates are internally independent (finite-product independence transferred to hyperfinite products).

\subsection{Matching stage: internal partial matching conditional on $h_n$}
\label{app:matching-stage}
\noindent\textbf{Input:} post-mutation type profile $h_n:I\to S$.

\medskip
\noindent\textbf{Goal:} construct an internal random partial matching $\pi_n:I\to I\cup\{J\}$ such that:
\begin{itemize}
  \item conditional on $h_n(i)=k$, $P(\pi_n(i)=J)=q_k$;
  \item conditional on being matched, partner types follow the ``proportional to matched mass'' rule (DS2007 equation (3));
  \item within a period, each agent is matched to at most one partner (partial matching).
\end{itemize}

\medskip
\noindent\textbf{Encoding trick (fixed points = no match).}
Instead of sampling $\pi_n$ directly, DS2007 effectively samples an internal bijection $\omega:I\to I$ such that:
\begin{itemize}
  \item $\omega(i)=i$ indicates \emph{no match} (a fixed point),
  \item if $\omega(i)\neq i$ then $\omega(\omega(i))=i$ (two-cycles encode pairs).
\end{itemize}
Given such an $\omega$, define
\[
\pi_n(i)\coloneqq
\begin{cases}
J, & \omega(i)=i,\\
\omega(i), & \omega(i)\neq i.
\end{cases}
\]
This ensures that within period $n$ each agent is either unmatched or matched to exactly one partner.

\medskip
\noindent\textbf{How to choose the unmatched set with type-dependent probabilities.}
For each $k\in S$, let
\[
A_k \coloneqq \{i\in I:\ h_n(i)=k\}\in \mathcal I_0
\]
be the internal set of post-mutation type-$k$ agents. DS2007 choose internal integers $m_k$ with
\[
\frac{m_k}{|A_k|}\approx q_k
\]
(infinitely close), and then choose internal subsets $B_k\subseteq A_k$ with $|B_k|=m_k$.
Intuitively, $B_k$ is the set of type-$k$ agents designated ``unmatched.''

\medskip
\noindent\textbf{Uniform matching on the matched pool.}
Let
\[
I^{\mathrm{match}}\coloneqq I\setminus \bigcup_{k\in S} B_k.
\]
On $I^{\mathrm{match}}$ (which is internally even-sized after suitable choice of $(m_k)$), sample a uniform random perfect matching (a fixed-point-free involution) and extend it by fixing each $i\in \bigcup_k B_k$.
This yields an internal distribution over partial matchings consistent with $(q_k)$ and with the proportional-to-matched-mass partner-type rule.

\subsection{Partner-type process and type-change stage}
\label{app:type-change-stage}
With $h_n$ and $\pi_n$ defined, set the partner-type process
\[
g_n(i)\coloneqq h_n(\pi_n(i))\in S\cup\{J\},
\]
using the convention $h_n(J)=J$.

\medskip
\noindent\textbf{Goal:} sample end-of-period types $a_n:I\to S$ such that:
\begin{itemize}
  \item if $g_n(i)=J$ then $a_n(i)=h_n(i)$ (no change when unmatched);
  \item if $g_n(i)=\ell\in S$ then $a_n(i)$ is distributed as $v_{h_n(i),\ell}$;
  \item conditional on $(h_n,g_n)$, the family $\{a_n(i)\}_{i\in I}$ is internally independent.
\end{itemize}

\medskip
\noindent\textbf{Construction (another product draw on $S^I$).}
For each agent $i$, define a probability measure $\xi_i$ on $S$ by:
\[
\xi_i \coloneqq
\begin{cases}
\delta_{h_n(i)}, & g_n(i)=J,\\
v_{h_n(i),g_n(i)}, & g_n(i)\in S.
\end{cases}
\]
Let $Q^{(3n)}_0\coloneqq \bigotimes_{i\in I}\xi_i$ be the internal product measure on $S^I$.
Draw $\omega^{(3n)}\sim Q^{(3n)}_0$ and set $a_n(i)\coloneqq \omega^{(3n)}(i)$.

\subsection{Why this stage-wise construction delivers ``conditional independence''}
\label{app:why-conditional-independence}
At the internal level, mutation and type-change stages are built from product measures on $S^I$, so conditional independence across agents is straightforward.
The matching stage is subtler because matching is without replacement, but DS2007 show (via explicit hyperfinite counting bounds) that the dependence induced by the matching constraint vanishes in the Loeb ``essential'' sense (independence holds for $\lambda$-a.e.\ pair of agents), yielding their Markov conditional independence in types.

\section{Appendix: Gluing across time via Loeb transition probabilities and generalized Ionescu--Tulcea}
\label{app:ionescu-tulcea}

The stage-wise constructions above specify, at each step, an internal \emph{transition probability} from the past to the next randomization outcome (a kernel).
To obtain one global probability space supporting \emph{all} periods simultaneously, DS2007 build an infinite product of Loeb transition probabilities.

\subsection{From internal transition probabilities to Loeb transition probabilities}
\label{app:loeb-transition}
An \emph{internal transition probability} assigns to each internal state $x$ an internal probability measure on a hyperfinite sample space.
Loeb-izing these state-dependent internal measures yields a family of standard (Loeb) measures $\{P_x\}$, called a \emph{Loeb transition probability} in DS2007.

\subsection{Generalized Fubini for Loeb transition probabilities (DS2007, Thm.\ 5.1)}
\label{app:gen-fubini}
DS2007 Theorem~5.1 is a Fubini-type theorem for the Loeb product $\lambda\boxtimes P$ when $P$ is a \emph{transition probability} (state-dependent family of measures) rather than a single fixed marginal.
This is the technical tool that allows iterated integrals and section measurability in the presence of state-dependent kernels.

\subsection{Generalized Ionescu--Tulcea for an infinite sequence of Loeb transition probabilities (DS2007, Thm.\ 5.5)}
\label{app:gen-ionescu}
Given a sequence of (Loeb) transition probabilities describing successive stages, DS2007 Theorem~5.5 constructs a unique countably additive probability measure on the infinite product $\sigma$-algebra that has the specified one-step kernels as its conditional distributions.
This is the infinite-horizon analog of the classical Ionescu--Tulcea theorem for Markov kernels, adapted to the Loeb setting.

\subsection{Resulting Fubini extension on the infinite product (DS2007, Prop.\ 5.9)}
\label{app:fubini-extension-infinite}
Finally, DS2007 Proposition~5.9 shows that the resulting infinite product space, equipped with the Loeb product measure, is again a Fubini extension of the corresponding usual product.
This ensures the global space is rich enough (measurability) and still supports the Fubini manipulations needed to apply ELLN stage-by-stage across all periods.

\end{document}


